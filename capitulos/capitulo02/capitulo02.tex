%*******************************************************
% Capitulo dos
%*******************************************************

\chapter{Marco teórico y estado del arte}

El Marco Teórico es el conocimiento mínimo necesario que se requiere para comprender el problema de investigación. La base teórica de referencia es la que permite comprender el problema y sus principales aspectos de detalle en toda su extensión. Las áreas principales que conciernen a ésta investigación son Interacción Humano-Máquina \cite{sigchi1992curricula}, Arquitectura Empresarial y Arquitectura de Software.

\section{Interacción Humano Máquina}

Según \cite{sigchi1992curricula} \footnote{Documento publicado en 1992 pero en su página web tiene fecha de actualización de 2009-07-29. http://old.sigchi.org/cdg/index.html} es una disciplina alrededor del diseño, evaluación e implementación de sistemas de computo interactivo para uso humano y el estudio de los fenómenos alrededor de esa interacción.

Los problemas clásicos estudiados contemplan la interacción de los humanos en estaciones de trabajo utilizando dispositivos de entrada/salida tradicionales como pantallas y teclados. Los problemas actuales incluyen toda gama de dispositivos, tamaños de sistemas, medios de presentación, la ubicuidad y la transformación ambiental.

En la literatura se encuentran numerosas referencias a la palabra \textit{machine} y su acepción cubre cualquier elemento de procesamiento donde se realice cómputo. Así, se puede analizar la interacción entre humanos y por ejemplo en dispositivos como datáfonos, tarjetas con chip, teléfonos celulares, tabletas, estaciones de trabajo tradicionales, electrodomésticos en el hogar hasta espacios inteligentes. El área de HCI (\textit{Human Computer Interaction}) es un área de estudio interdisciplinar de la ingeniería en donde las ciencias de la computación y el diseño industrial son muy importantes y transdisciplinar con áreas como la psicología, la sociología y la antropología.

\subsection{Arquitectura de interfaces y Sistemas de computo}

Según \cite{sigchi1992curricula} uno de los elementos en los que se divide el estudio de HCI es la Arquitectura de interfaces y Sistemas de computo\footnote{Título de capítulo: \textit{Computer System and Interface Architecture}, Sec. 2.3.4. Unidad C. Pag 22. }, en la cual se pueden distinguir: dispositivos de entrada y salida, técnicas de interacción, tipos de interacción, gráficas por computador, arquitectura de la interacción.

\subsubsection{Dispositivos de entrada y salida}

Esta sub-área de conocimiento se encarga de los detalles técnicos de los dispositivos de entrada y salida entre humanos y máquinas. Los dispositivos de entrada son todos los posibles mecanismos por los cuales información que es generada por los humanos, de manera explicita o implícita, llega a los sistemas de cómputo. Algunos ejemplos pueden ser diferentes tipos de cámaras en combinación con mecanismos específicos (como seguimiento y reconocimiento del dedo índice o los ojos). De manera análoga, los mecanismos de salida incluyen todos los mecanismos de presentación de información visual, sonora o física. En los mecanismos de salida también se incluyen los actuadores sobre dispositivos hardware que no sean pantallas.

\subsubsection{Técnicas de interacción}

Este subconjunto del área de conocimiento describe todas las técnicas y arquitecturas básicas para la interacción con los humanos.

En interacciones de entrada podemos encontrar diferentes tipos de propósitos como seleccionar, especificar parámetros o controlar de manera continua alguna variable. Las técnicas sobre las entradas se pueden dividir en técnicas basadas en teclado, mouse, lápiz o voz. En interacciones de salida se pueden encontrar todos lo propósitos de salida, usualmente relacionados con transmitir, resumir, ilustrar o visualizar información. No descrito por el \textit{Curricula  for  human-computer  interaction} están también todas las formas de material multimedia y juegos en los cuales el componente multimedia es el más relevante. Las técnicas asociadas con la salida de información incluyen desplazamiento de pantalla, diagramación de ventanas, animaciones y proyecciones en espacios no tradicionales.

Output techniques (e.g., scrolling display, windows, animation, sprites, fish-eye displays)
Screen layout issues (e.g., focus, clutter, visual logic)
Dialogue Interaction Techniques:
Dialogue type and techniques (e.g., alphanumeric techniques, form filling, menu selection, icons and direct manipulation, generic functions, natural language)
Navigation and orientation in dialogues, error management
Multimedia and non-graphical dialogues: speech input, speech output, voice mail, video mail, active documents, videodisc, CD-ROM
Agents and AI techniques
Multi-person dialogues
Dialogue Issues:
Real-time response issues
Manual control theory
Supervisory control, automatic systems, embedded systems
Standards
"Look and feel," intellectual property protection
C3. Dialogue Genre {p. 24}

The conceptual uses to which the technical means are put. Such concepts arise in any media discipline (e.g., film, graphic design, etc.).

Interaction metaphors (e.g., tool metaphor, agent metaphor)
Content metaphors (e.g., desktop metaphor, paper document metaphor)
Persona, personality, point of view
Workspace models
Transition management (e.g., fades, pans)
Relevant techniques from other media (e.g., film, theater, graphic design)
Style and aesthetics
C4. Computer Graphics {p. 24}

Basic concepts from computer graphics that are especially useful to know for HCI.

Geometry in 2- and 3- space, linear transformations
Graphics primitives and attributes: bitmap and voxel representations, raster-op, 2-D primitives, text primitives, polygon representation, 3-D primitives, quadtrees and octtrees, device independent images, page definition languages
Solid modeling, splines, surface modeling, hidden surface removal, animation, rendering algorithms, lighting models
Color representation, color maps, color ranges of devices
C5. Dialogue Architecture {p. 25}

Software architectures and standards for user interfaces.

Layers model of the architecture of dialogues and windowing systems, dialogue system reference models
Screen imaging models (e.g., RasterOp, Postscript, Quickdraw)
Window manager models (e.g., Shared address-space, client-server), analysis of major window systems (e.g., X, New Wave, Windows, Open Look, Presentation Manager, Macintosh)
Models of application-to-dialogue manager connection
Models for specifying dialogues
Multi-user interface architectures "Look and feel"
Standardization and interoperability

\section{Estado del arte}

El Estado del Arte es el conocimiento necesario más actualizado que existe para resolver el problema de investigación planteado y se compone de todos los conocimientos e investigaciones más recientes que han formulado una solución al problema de investigación o han contribuido sustancialmente con algún aspecto de la solución del mismo.

El Estado del Arte constituye la base más profunda de la investigación científica que permite descubrir conocimiento nuevo al revisar la literatura asociada al tema de investigación de manera que pueda determinarse quienes, cómo, cuándo, dónde y por qué han tratado de resolver el problema de investigación planteado, determinar su actualización y verificar si el tema sigue vigente así como descubrir hasta donde ha avanzado el conocimiento admisible mas reciente sobre el tema planteado.

El estudio profundo del Estado del Arte permite identificar rápidamente la frontera del conocimiento respecto al problema de investigación y eso significa que cualquier desviación y aspecto por estudiar a partir del estado del arte conduce casi directamente al desarrollo de  conocimiento nuevo para los investigadores.
En ese contexto la lectura de documentos y material de contenido científico que permite revisar el Estado del Arte de los temas de investigación ha sido tratada con mayor detalle en el artículo http://max-schwarz.blogspot.com/2012/12/como-leer-un-paper-de-investigacion.html  donde se precisan las secciones, partes y componentes de los documentos y la manera como deben comprenderse para que los investigadores puedan sacarle el máximo provecho.
