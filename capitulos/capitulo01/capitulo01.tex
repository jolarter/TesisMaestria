%*******************************************************
% Capitulo uno
%*******************************************************

\chapter{Introducción}

Este trabajo propone el montaje de una instalación interactiva como caso de estudio en la aplcación del paradigma de Computación Pervasiva\cite{RN1}, analizando diferentes elementos como impacto, áreas de trabajo, proceso de construcción, proceso creativo, articulación con otras áreas de ingeniería y otras facetas que permitan su aprovechamiento como una tecnología útil en espacios artísticos. Como tema de estudio principal en la aplicación de la tecnología se usa la reconstrucción de memoria histórica del conflicto armado en Colombia y como eje principal, la magnitud del fenómeno de desaparición forzada. En este capítulo se describen brevemente los aspectos básicos del proyecto, sus objetivos, alcances y una comprensión básica del problema.    

\section{Presentación}

La reconstrucción de memoria histórica es uno de los principales mecanismos para la comprender las razones del origen y transformación del conflicto armado interno colombiano, sus diferentes actores y las vivencias de las víctimas que han sufrido de supresión y silenciamento. 

El Centro Nacional de Memoria Histórica (CNMH)\footnote{Centro Nacional de Memoria Histórica, http://www.centrodememoriahistorica.gov.co} es una institución adscrita al Departamento para la Prosperidad Social cuyo objetivo principal es contribuir a la realización de la reparación integral y el derecho a la verdad del que son titulares las víctimas. Bajo este objeto social se realizan actividades permanentes de construcción de memoria, destacando una gestión de iniciativas de memoria en las que el pueblo colombiano participa de forma permanente y autónoma. Al igual que esta institución muchas otras intituciones públicas e iniciativas privadas usan el arte como uno de los mecanismos para reconstruir y procesar elementos de la memoria histórica del conflicto armado. Uno de los referentes más importantes para este trabajo es el inventario de iniciativas artísticas gestionado por el CNMH. 

Como objeto de estudio de éste trabajo de grado se analizan las instalaciones interactivas que pudieran ser construidas con elementos de cómputo específicos en una combinación de arte plástico digital o interactivo. Existen numerosos ejemplos
\cite{RN35,RN25,RN34,RN38,RN31,RN29,RN40,RN37}
de creaciones digitales en las que el arte es el eje central y en las que la transformación del espacio a través de interfaces humano-máquina sale de esquemas tradicionales. También existen desde las artes liberales y las ciencias humanas investigaciones acerca de paz, guerra, memoria y arte, que permiten entender las diferentes aplicaciones del arte en temas de paz \cite{Jimenez201672,SotoAguilar201591,EstripeautBourjac2013154}. 

La computación pervasiva, describe una rama en las ciencias de la computación que investiga la creación de ambientes saturados de cómputo invisible, en el que las interfaces humano-máquina se extienden a muchos tipos de sensores. Según \cite{RN1}, la visión original de la computación pervasiva describe que los artefactos de un entorno serán fabricados de tal manera que los incrementos tecnológicos sean indistinguibles del objeto mismo. En este sentido el cómputo podrá impregnar de forma invisible la mayoría de objetos en un entorno y podrían recolectar información de forma transparente al usuario para después entregar respuestas en tiempo real o de forma natural.  Algunos de los retos y aspectos más importantes en la computación pervasiva son los espacios inteligentes, la movilidad, el computo distribuido, la invisibilidad, la escalabilidad local y la respuesta a eventos heterogéneos; todos aplicables al diseño de espacios artísticos con la ventaja de ser espacios que permiten una alta experimentación.

Como parte de este trabajo, se consultará y se recibirá apoyo de expertos en otras disciplinas fuera de la ingeniería de software, las interfaces humano-máquina y en general las ciencias de la computación; principalmente los temas de creación artística, instalaciones artísticas digitales, memorias del conflicto armado y desaparición forzada. Este trabajo cuenta con el apoyo del semillero de investigación en participación política y el consultorio de psicología, ambos del Politécnico Grancolombiano.

Bajo estos tres ejes principales y interdisciplinares: la computación pervasiva, la creación artística digital y la reconstrucción de memoria histórica; se presenta una propuesta de construcción de espacios de arte interactivo con un propósito social especifico del contexto del posconficto colombiano, que permite analizar la aplicabilidad de una tecnología que tiene el fin de transformar los espacios y los objetos con los que los humanos interactúan y en la que el entorno de aplicación en una situación que hace parte de la problemática y la situación actual de Colombia.


\section{Justificación}

La computación pervasiva incorpora diferentes áreas de conocimiento como interacción humano máquina, sistemas distribuidos, la computación móvil, los espacios inteligentes, la computación invisible; todas ellas con el fin de crear espacios en los que el cómputo permea el entorno humano de forma invisible e inteligente y en donde en el futuro los objetos y sus incrementos tecnológicos son indistinguibles. Investigaciones como [][][][] muestran casos de aplicación y frameworks que permiten comprender y experimentar la computación ubicua pero ninguna contempla el estudio desde la ingeniería de aplicaciones asociadas al arte digital. Por otro lado, publicaciones respecto de arte digital[][] presentan de una forma rica experiencias de construcción de instalaciones digitales pero su mirada no contempla tampoco el análisis desde el tema de computación ubicua. 

Además, la aplicación especifica de ésta tecnología sobre temas de reconstrucción de memoria es nula en cuanto a revisión bibliográfica. Por tanto, este trabajo pretende ampliar la base de investigación de la computación ubicua aplicada a áreas especificas de conocimiento, y en particular, al tema de reconstrucción de memoria del conflicto armado colombiano.   

\section{Objetivos}

\subsection{Objetivo principal}

Analizar las dinámicas de la aplicación de la computación pervasiva al diseño y montaje de una instalación artística sobre reconstrucción de memoria histórica y desaparición forzada.

\subsection{Objetivos específicos}

\begin{itemize}

    \item Recopilar información de contexto acerca de reconstrucción de memoria histórica y procesarla para incorporarla en un sistema de computo ubicuo.
    
    \item Analizar las metodologías de investigación-creación versus metodologías ágiles en la construcción de software para el propósito específico de éste trabajo de grado.

    \item Diseñar y construir un sistema software aplicando los conceptos de computación pervasiva.

    \item Realizar un montaje y presentación de la instalación artística junto con un análisis de su desempeño, funcionalidad, integración y usabilidad.

\end{itemize}

\section{Línea base cronograma}

\begin{enumerate}
	\item \label{puntouno} Recolección de información fuente acerca del conflicto armado colombiano a partir de la información del Centro de Memoria Histórica.
	\item \label{puntodos} Recolección de información de referentes y material audiovisual acerca de instalaciones artísticas que tengan componentes digitales.
	\item \label{puntotres} Procesamiento de la información de memoria histórica y su adaptación a sistemas ubicuos.
	\item \label{puntocuatro} Ciclos de diseño y desarrollo incrementales del software a usar en el montaje.
	\item \label{puntocinco} Montaje de cosas
	
	
	
	
\end{enumerate}

\definecolor{midgray}{gray}{.5}
\begin{table}[!htbp]
	\centering
		\begin{tabular}{|c|c|c|c|c|c|c|c|c|c|c|}
		\hline
		&\multicolumn{5}{c|}{2010}&\multicolumn{5}{c|}{2010}\\
		\hline
		&MAR&ABR&MAI&JUN&JUL&AGO&SET&OUT&NOV&DEZ\\
		\hline
		\ref{ela-pro}&\cellcolor{midgray}&&&&&&&&&\\
		\hline
		\ref{anI}&&\cellcolor{midgray}&&&&&&&&\\
		\hline	
		\ref{anII}&&\cellcolor{midgray}&&&&&&&&\\
		\hline			
		\ref{anIII}&&\cellcolor{midgray}&\cellcolor{midgray}&&&&&&&\\
		\hline	
		\ref{dI}&&&\cellcolor{midgray}&&&&&&&\\
		\hline
		\ref{dII}&&&\cellcolor{midgray}&\cellcolor{midgray}&&&&&&\\
		\hline	
		\ref{dIII}&&&&\cellcolor{midgray}&\cellcolor{midgray}&&&&&\\
		\hline	
		\ref{esc-tcI}&&&\cellcolor{midgray}&\cellcolor{midgray}&\cellcolor{midgray}&&&&&\\
		\hline	
		\ref{imI}&&&&&\cellcolor{midgray}&&&&&\\
		\hline	
		\ref{imII}&&&&&&\cellcolor{midgray}&&&&\\
		\hline	
		\ref{imIII}&&&&&&\cellcolor{midgray}&\cellcolor{midgray}&\cellcolor{midgray}&&\\
		\hline	
		\ref{tec}&&&&&&&&\cellcolor{midgray}&\cellcolor{midgray}&\\
		\hline	
		\ref{esc-tcII}&&&&&&&&\cellcolor{midgray}&\cellcolor{midgray}&\cellcolor{midgray}\\
		\hline	
		\end{tabular}
\end{table}

\section{Alcance y entregables}

Éste proyecto se limitará a las siguientes:

\begin{itemize}
    \item Se realizará construcción y montaje de los diferentes elementos de la instalación artística desde las actividades del desarrollo de software y el diseño de sistemas ubicuos. No se contempla el estudio de construcción o diseño artístico, semiología, antropología o alguna otra disciplina fuera de la ingeniería, pero se analizará como esos diferentes elementos pueden ser relacionados con la computación ubicua.
    \item Se usará la información publicada por el centro de memoria histórica como referente principal. No se hará construcción o búsqueda de información de antecedentes históricos como parte de éste trabajo.
    \item No se realizarán estudios acerca de la dirección, escenografía, estética o actuación. Pero se analizará como esos diferentes elementos se pueden relacionar con la computación ubicua.
\end{itemize}

Al final de éste trabajo se espera:

\begin{itemize}
\item Información de reconstrucción de memoria histórica procesada de tal manera que pueda ser incorporada como parte de un sistema de computación ubicua en un espacio artístico. En éste proceso además es necesario encontrar mecanismos adecuados para representación de información bajo la filosofía de la tecnología propuesta.  
\item Los diseños y la construcción de un sistema software distribuido, móvil, invisible e inteligente para computo ubicuo para la instalación artística.
\item Un montaje y presentación de la instalación artística junto con el análisis del proceso y la experiencia. 
\end{itemize}

\section{Metodología}

El proceso de desarrollo del proyecto tiene tres etapas principales:
\begin{itemize}
    \item Recolección de referentes fundamentales y análisis de la información del conflicto armado colombiano. Esta fase deberá realizarse en acompañamiento de otras disciplinas. Además deberá realizarse un análisis de mecanismos de representación de la información escogida bajo la tecnología propuesta.
    
    \item Análisis, diseño y construcción de un sistema ubicuo para el problema propuesto. En esta fase solo habrá un desarrollador de software pero se utilizarán características de modelos de proceso iterativos e incrementales en coordinación con el diseño artístico.
    \item Montaje de la instalación artística. Este proceso incluye la caracterización del montaje y, posterior a la ejecución, documentación de la experiencia.
\end{itemize}

