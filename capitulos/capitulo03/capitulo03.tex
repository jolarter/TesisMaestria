%*******************************************************
% Capitulo tres
%*******************************************************

\chapter{Estado del arte}

El Estado del Arte es el conocimiento necesario más actualizado que existe para resolver el problema de investigación planteado y se compone de todos los conocimientos e investigaciones más recientes que han formulado una solución al problema de investigación o han contribuido sustancialmente con algún aspecto de la solución del mismo. 

El Estado del Arte constituye la base más profunda de la investigación científica que permite descubrir conocimiento nuevo al revisar la literatura asociada al tema de investigación de manera que pueda determinarse quienes, cómo, cuándo, dónde y por qué han tratado de resolver el problema de investigación planteado, determinar su actualización y verificar si el tema sigue vigente así como descubrir hasta donde ha avanzado el conocimiento admisible mas reciente sobre el tema planteado.

El estudio profundo del Estado del Arte permite identificar rápidamente la frontera del conocimiento respecto al problema de investigación y eso significa que cualquier desviación y aspecto por estudiar a partir del estado del arte conduce casi directamente al desarrollo de  conocimiento nuevo para los investigadores.
En ese contexto la lectura de documentos y material de contenido científico que permite revisar el Estado del Arte de los temas de investigación ha sido tratada con mayor detalle en el artículo http://max-schwarz.blogspot.com/2012/12/como-leer-un-paper-de-investigacion.html  donde se precisan las secciones, partes y componentes de los documentos y la manera como deben comprenderse para que los investigadores puedan sacarle el máximo provecho.
