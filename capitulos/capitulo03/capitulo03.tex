%*******************************************************
% Capitulo tres
%*******************************************************

\chapter{Caso de estudio}
\label{Caso_de_estudio}

Se propone la intervención artística 'No los olvidamos' como caso de estudio para este trabajo de grado. La intervención hace parte del trabajo del colectivo artístico 'Aitawa: Reconstrucción de memoria histórica con arte'. A continuación se describe el panorama general de la obra, sus participantes, los elementos digitales o interactivos que incorpora y su metodología de construcción.

\section{Antecedentes, proposito y resultados}

Como parte de su misión, entidades gubernamentales como la 'Unidad para las víctimas' o la secretaría de Gobierno con su 'Alta consejería para las victimas', desarrollan procesos permanentes de reparación integral para las victimas del conflicto para que ejerzan su ciudadanía y aporten a la consolidación de la paz. Como resultado de sus acciones se han realizado talleres en el pasado que permiten la reunión de diferentes grupos de víctimas y que éstos grupos creen vinculos sociales activos. Un grupo de éstas personas se ha reunido y ha creado un colectivo para desarrollar acciones artísticas que les permitan construir lazos entre ellos, hacer catarsis de las diferentes vivencias dentro del conflicto armado, hacer visibles muchas de sus inquietudes y necesidades, desarrollar medidas de reconstrucción simbólica y de memoria y además identificar entidades que apoyen su gestión.

Una de las acciones del colectivo fue el desarrollo de una intervención artística llamada 'No los olvidamos', la cual consistió en el desarrollo de varios talleres con población inscrita en el registro único de víctimas y en la que se desarrollaban diferentes temas acerca de la reconstrucción simbólica de la memoria de algunas de las vivencias de los paticipantes. El proceso se orientó principalmente al flagelo de desaparición forzada y contó con el apoyo del consultorio de psicología y del Semillero en participación política, ambos del Politécnico Grancolombiano.

Durante los talleres se realizaron pequeñas piezas de origami que los participantes usaban para representar algunas de sus memorias y se realizaban guias y talleres desarrollados por los miembros del Colectivo\footnote{Los talleres y las guías no hacen parte de éste trabajo de grado y son propiedad de sus respectivos autores.}. Algunos de los talleres incluyeron acciones que permitieron utilizar la metodolgía de construcción colectiva, la cual es una metodología usada para creación artística y cuyos resultados se usaron para la aplicación de la tecnología de computación pervasiva.

Todo el proceso fue acompañado por Idartes en el proceso de participación ciudadana de construcción del Libro al Viento 2018, para el capítulo de poblaciones el cuál será escrito por Margarita XXXXX quién participó en el último taller que el colectivo realizó.

\section{Construcción}

Todos los talleres desarrollados prermitieron planear y construir una obra plástica digital e interactiva. La obra se propone como el taller final de la intervención artística y la cuál se incluye lo desarrollado en talleres anteriores. La obra consta de tres espacos físicos; el primero de los espacios es llamado 'Presencia', el segundo 'ausencia' y el último 'Duelo'.

La presentación de la obra ocurre en el es patrocinada por xxxxx

recibe el apoyo de xxxxx

incorpora los siguientes elementos digitales xxxxx

fue expuesta en xxxxxxx

El caso de estudio será desarrollado con la metodología xxxxxxx de creación artística

Descripcioón de la estructura.


\section{Talleres de construcción}


\section{Los espacios}


\subsection{Presencia}

Es el inicio de la Instalación. El espacio consta de un mural de recostrucción simbolica de la presencia de personas desaparecidas a partir de l

\subsection{Ausencia}

\subsection{Duelo}

\section{Componentes digitales}

\subsection{Espacio de memoria}

\subsection{El Portal}
