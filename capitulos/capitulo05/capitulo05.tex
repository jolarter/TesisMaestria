%*******************************************************
% Capitulo cinco
%*******************************************************
\chapter{Diseño y construcción del software}

La intervención artística cuenta con dos elementos digitales que deben ser trabajados en conjunto: \textit{El portal} y \textit{El espacio de memoria}.

El proceso de intervención, al mismo tiempo que el proceso de producción, comienzan con las reuniones de consolidación y declaración del proyecto, posteriormente continuados con los talleres; hubo seis talleres de trabajo con las personas de grupo que aportaron a la obra y dos reuniones más para inicio y consolidación.

El proceso de intervención artística inicia desde la primera reunión y va hasta la presentación de la muestra en la cual el software está listo para interactuar con el público.

A continuación se describen las iteraciones y los detalles que permitieron diseñar y construir los componentes de software.

\section{Iteraciones}

Las iteraciones, o Sprints, ocurren desde la reunión del grupo y finalizan antes de la siguiente reunión. Las reuniones funcionan como el \textit{Sprint planning meeting} dentro de la metodología modificada y permiten refinar el diseño de las piezas y componentes. Varias pruebas ocurrieron en las reuniones utilizando los dispositivos kinect para que los miembros del equipo pudieran entender como era su funcionamiento. En cada uno de los incrementos se modificaba el \textit{product backlog} y se producian entregas de valor en forma de prototipos funcionales con lo que los miembros del equipo podían interactuar. 
