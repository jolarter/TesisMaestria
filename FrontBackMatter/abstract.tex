%*******************************************************
% Abstract
%*******************************************************
\pdfbookmark[1]{Abstract}{Abstract}
\chapter*{Resumen}

La reconstrucción de memoria histórica es uno de los principales mecanismos para la comprender las razones del origen y transformación del conflicto armado interno colombiano, sus diferentes actores y las vivencias de las víctimas que han sufrido de supresión y silenciamento. Uno de los mecanismos para reconstruir y procesar elementos de la memoria histórica del conflicto armado y comprender su magnitud es el arte y como objeto de estudio de éste trabajo, las instalaciones artísticas/interactivas que pudieran ser construidas con elementos de cómputo específicos en una combinación de arte digital. La computación pervasiva\footnote{Se propone el anglicismo \textit{'Computación Pervasiva'} que proviene de la frase \textit{'Pervasive Computation'} como descripción de la tecnología que será aplicada en éste trabajo de grado. Las traducciones de la palabra  \textit{pervasive} al español no proporcionan una acepción correcta de su significado en ingles y podrían distraer al lector de su significado original.}, describe un paradigma en las ciencias de la computación que investiga la creación de ambientes saturados de cómputo invisible, en el que las interfaces humano-máquina se extienden a muchos tipos de sensores y pueden responder a necesidades humanas basándose en espacios inteligentes, movilidad, computo distribuido, invisibilidad, escalabilidad local y respuesta a eventos heterogéneos. Este trabajo propone el montaje de una instalación interactiva como caso de estudio de aplicación de la computación pervasiva, teniendo como tema principal la reconstrucción de memoria histórica y la desaparición forzada; analizando su impacto, áreas de trabajo, proceso de construcción, proceso creativo, articulación con otras áreas de la ingeniería y de las ciencias humanas, y otras facetas que permitan su aprovechamiento, distribución y producción. 
